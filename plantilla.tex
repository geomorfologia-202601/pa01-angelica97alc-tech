% Options for packages loaded elsewhere
\PassOptionsToPackage{unicode}{hyperref}
\PassOptionsToPackage{hyphens}{url}
%
\documentclass[
]{article}
\usepackage{amsmath,amssymb}
\usepackage{iftex}
\ifPDFTeX
  \usepackage[T1]{fontenc}
  \usepackage[utf8]{inputenc}
  \usepackage{textcomp} % provide euro and other symbols
\else % if luatex or xetex
  \usepackage{unicode-math} % this also loads fontspec
  \defaultfontfeatures{Scale=MatchLowercase}
  \defaultfontfeatures[\rmfamily]{Ligatures=TeX,Scale=1}
\fi
\usepackage{lmodern}
\ifPDFTeX\else
  % xetex/luatex font selection
\fi
% Use upquote if available, for straight quotes in verbatim environments
\IfFileExists{upquote.sty}{\usepackage{upquote}}{}
\IfFileExists{microtype.sty}{% use microtype if available
  \usepackage[]{microtype}
  \UseMicrotypeSet[protrusion]{basicmath} % disable protrusion for tt fonts
}{}
\makeatletter
\@ifundefined{KOMAClassName}{% if non-KOMA class
  \IfFileExists{parskip.sty}{%
    \usepackage{parskip}
  }{% else
    \setlength{\parindent}{0pt}
    \setlength{\parskip}{6pt plus 2pt minus 1pt}}
}{% if KOMA class
  \KOMAoptions{parskip=half}}
\makeatother
\usepackage{xcolor}
\usepackage[margin=1in]{geometry}
\usepackage{longtable,booktabs,array}
\usepackage{calc} % for calculating minipage widths
% Correct order of tables after \paragraph or \subparagraph
\usepackage{etoolbox}
\makeatletter
\patchcmd\longtable{\par}{\if@noskipsec\mbox{}\fi\par}{}{}
\makeatother
% Allow footnotes in longtable head/foot
\IfFileExists{footnotehyper.sty}{\usepackage{footnotehyper}}{\usepackage{footnote}}
\makesavenoteenv{longtable}
\usepackage{graphicx}
\makeatletter
\def\maxwidth{\ifdim\Gin@nat@width>\linewidth\linewidth\else\Gin@nat@width\fi}
\def\maxheight{\ifdim\Gin@nat@height>\textheight\textheight\else\Gin@nat@height\fi}
\makeatother
% Scale images if necessary, so that they will not overflow the page
% margins by default, and it is still possible to overwrite the defaults
% using explicit options in \includegraphics[width, height, ...]{}
\setkeys{Gin}{width=\maxwidth,height=\maxheight,keepaspectratio}
% Set default figure placement to htbp
\makeatletter
\def\fps@figure{htbp}
\makeatother
\setlength{\emergencystretch}{3em} % prevent overfull lines
\providecommand{\tightlist}{%
  \setlength{\itemsep}{0pt}\setlength{\parskip}{0pt}}
\setcounter{secnumdepth}{-\maxdimen} % remove section numbering
% definitions for citeproc citations
\NewDocumentCommand\citeproctext{}{}
\NewDocumentCommand\citeproc{mm}{%
  \begingroup\def\citeproctext{#2}\cite{#1}\endgroup}
\makeatletter
 % allow citations to break across lines
 \let\@cite@ofmt\@firstofone
 % avoid brackets around text for \cite:
 \def\@biblabel#1{}
 \def\@cite#1#2{{#1\if@tempswa , #2\fi}}
\makeatother
\newlength{\cslhangindent}
\setlength{\cslhangindent}{1.5em}
\newlength{\csllabelwidth}
\setlength{\csllabelwidth}{3em}
\newenvironment{CSLReferences}[2] % #1 hanging-indent, #2 entry-spacing
 {\begin{list}{}{%
  \setlength{\itemindent}{0pt}
  \setlength{\leftmargin}{0pt}
  \setlength{\parsep}{0pt}
  % turn on hanging indent if param 1 is 1
  \ifodd #1
   \setlength{\leftmargin}{\cslhangindent}
   \setlength{\itemindent}{-1\cslhangindent}
  \fi
  % set entry spacing
  \setlength{\itemsep}{#2\baselineskip}}}
 {\end{list}}
\usepackage{calc}
\newcommand{\CSLBlock}[1]{\hfill\break\parbox[t]{\linewidth}{\strut\ignorespaces#1\strut}}
\newcommand{\CSLLeftMargin}[1]{\parbox[t]{\csllabelwidth}{\strut#1\strut}}
\newcommand{\CSLRightInline}[1]{\parbox[t]{\linewidth - \csllabelwidth}{\strut#1\strut}}
\newcommand{\CSLIndent}[1]{\hspace{\cslhangindent}#1}
\ifLuaTeX
  \usepackage{selnolig}  % disable illegal ligatures
\fi
\usepackage{bookmark}
\IfFileExists{xurl.sty}{\usepackage{xurl}}{} % add URL line breaks if available
\urlstyle{same}
\hypersetup{
  pdftitle={¿Siguen vigentes los clasicos? Una revision critica de los modelos clasicos},
  pdfauthor={Angélica Maria Alcántara Ogando (Matrícula FI3276)},
  hidelinks,
  pdfcreator={LaTeX via pandoc}}

\title{¿Siguen vigentes los clasicos? Una revision critica de los modelos clasicos}
\author{Angélica Maria Alcántara Ogando (Matrícula FI3276)}
\date{2026-02-14}

\begin{document}
\maketitle

{
\setcounter{tocdepth}{2}
\tableofcontents
}
\section{Resumen (Abstract)}\label{resumen-abstract}

\textbf{Palabras clave:} Geomorfología; Ciclo Geográfico; Davis; Penck; Evolución del Relieve.

\section{1. Introducción}\label{introducciuxf3n}

\subsection{1.1. Contextualización}\label{contextualizaciuxf3n}

Definición de la transición de la geomorfología histórica (descriptiva) hacia la geomorfología de procesos (sistémica).

\subsection{1.2. Problema y justificación}\label{problema-y-justificaciuxf3n}

La persistencia de modelos del siglo XIX, como el de Davis, en textos fundamentales como \textbf{Tarbuck \& Lutgens (2005)} a pesar de las críticas geofísicas.

\subsection{1.3. Preguntas de investigación}\label{preguntas-de-investigaciuxf3n}

\begin{itemize}
\tightlist
\item
  \textbf{P1.} ¿Por qué los modelos clásicos mantienen su valor educativo pese a su declive científico?
\item
  \textbf{P2.} ¿Qué fallas metodológicas impiden el uso de estos modelos en la investigación de frontera actual?
\end{itemize}

\subsection{1.4. Objetivo}\label{objetivo}

Realizar una revisión crítica de los postulados de Davis y Penck y analizar su transición hacia los modelos de sistemas abiertos.

\section{2. Marco Teórico: Modelos de Evolución del Relieve}\label{marco-teuxf3rico-modelos-de-evoluciuxf3n-del-relieve}

\subsection{2.1. Antecedentes y Estado del Arte: De la Narrativa al Proceso}\label{antecedentes-y-estado-del-arte-de-la-narrativa-al-proceso}

La geomorfología, como disciplina científica, experimentó su primera gran sistematización a finales del siglo XIX con la propuesta del Ciclo Geográfico de William Morris Davis (Davis 1973). Durante décadas, este modelo descriptivo dominó la disciplina, ofreciendo una visión de la Tierra donde el paisaje evolucionaba de forma predecible a través de estadios de juventud, madurez y vejez (\textbf{tarbuck2020?}). En esta etapa, la geomorfología era fundamentalmente histórica y cualitativa.

Sin embargo, a mediados del siglo XX, surgieron críticas lideradas por la escuela alemana de Walther Penck (\textbf{simons1962morphological?}) y, posteriormente, por la Revolución Cuantitativa de Arthur Strahler (Strahler 1950). Estos autores cuestionaron el ``estatismo'' de Davis, argumentando que el relieve no es un sistema cerrado cuya energía se agota con el pasar del tiempo, sino un sistema abierto que responde constantemente a la tectónica y al clima (Cleverson, Schmidt, and Miller 2024). A pesar de que las investigaciones se desplazaron hacia el uso de modelos matemáticos y la geocronología, el interés por los modelos clásicos no ha desaparecido.

Como señala Luo (2024), las teorías de Davis han experimentado una ``reencarnación'' en el siglo XXI. Este nuevo interés no busca validar sus errores tectónicos, sino rescatar su valor como lenguaje común y herramienta pedagógica (Walsh and Vowles 2026). Por lo tanto, el debate entre ``clásicos'' y ``modernos'' no debe considerarse un capítulo terminado, sino una discusión necesaria para entender cómo la simplificación didáctica presente en textos base como el de (\textbf{tarbuck2020?}) convive con la complejidad de la investigación contemporánea (Orme 2007).

\begin{center}\rule{0.5\linewidth}{0.5pt}\end{center}

\subsection{2.2. Bases Teóricas: Los Modelos de Evolución del Relieve}\label{bases-teuxf3ricas-los-modelos-de-evoluciuxf3n-del-relieve}

\subsubsection{2.2.1. El Ciclo Geográfico de W.M. Davis}\label{el-ciclo-geogruxe1fico-de-w.m.-davis}

El modelo davisiano se fundamenta en la premisa de que el paisaje es una función de la estructura, el proceso y el tiempo (Davis 1973). El postulado central asume un levantamiento tectónico rápido e inicial de una masa de tierra, seguido de un prolongado periodo de estabilidad cortical donde la erosión actúa sin interrupciones tectónicas (\textbf{tarbuck2020?}).

Davis dividió esta evolución en estadios cronológicos, como se puede ver en la Figura @ref(fig:ciclo\_davis):

\begin{figure}
\includegraphics[width=1\linewidth]{ciclo-davis} \caption{Ciclo de erosión de Davis}(\#fig:fig-ciclo_davis)
\end{figure}

\begin{itemize}
\tightlist
\item
  \textbf{Juventud:} Caracterizada por valles estrechos en forma de ``V'' y una alta energía fluvial.
\item
  \textbf{Madurez:} Donde el relieve alcanza su máxima diferenciación y los valles se ensanchan.
\item
  \textbf{Vejez:} El estadio final donde la superficie se reduce a una \emph{peneplanicie}, una llanura casi sin rasgos situada cerca del nivel de base (Orme 2007).
\end{itemize}

\subsubsection{2.2.2. El Análisis Morfológico de Walther Penck}\label{el-anuxe1lisis-morfoluxf3gico-de-walther-penck}

Penck rechazó la idea de un levantamiento instantáneo, proponiendo que la erosión y la tectónica son procesos concurrentes. Su análisis se centra en la relación entre la intensidad del levantamiento y la tasa de degradación fluvial (\textbf{simons1962morphological?}).

A diferencia del ``suavizamiento'' de Davis, Penck introdujo el concepto de retroceso paralelo de las vertientes (\emph{slope replacement}). En este modelo, las pendientes escarpadas mantienen su ángulo mientras retroceden, siendo reemplazadas en la base por pendientes más suaves (Cleverson, Schmidt, and Miller 2024). Este proceso da lugar al \emph{Piedmonttreppen} o ``escalinata de piedemonte'', una serie de plataformas escalonadas que Penck interpretaba como evidencia de pulsos tectónicos de intensidad creciente (\textbf{simons1962morphological?}).

\subsubsection{2.2.3. Mecanismos Comparados}\label{mecanismos-comparados}

Para sintetizar las diferencias fundamentales entre ambos enfoques, se presenta la siguiente tabla comparativa que resume los supuestos críticos de cada autor.

\begin{center}\rule{0.5\linewidth}{0.5pt}\end{center}

\subsection{2.3. Definición de Conceptos Clave}\label{definiciuxf3n-de-conceptos-clave}

Para garantizar la precisión en la revisión crítica de los modelos, es imperativo establecer las definiciones técnicas de los conceptos que articulan la transición de la geomorfología histórica a la de procesos.

\subsubsection{\texorpdfstring{2.3.1. Peneplanicie (\emph{Peneplain})}{2.3.1. Peneplanicie (Peneplain)}}\label{peneplanicie-peneplain}

Concepto central del modelo davisiano que describe el estadio final de la degradación continental. Se define como:

\begin{quote}
``Una superficie de erosión de gran extensión que presenta un relieve ondulado muy suave, situada cerca del nivel de base, y que representa el límite teórico de la denudación tras un largo periodo de estabilidad tectónica'' (Davis 1973; en \textbf{tarbuck2020?}).
\end{quote}

\subsubsection{\texorpdfstring{2.3.2. \emph{Piedmonttreppen} (Escalinata de Piedemonte)}{2.3.2. Piedmonttreppen (Escalinata de Piedemonte)}}\label{piedmonttreppen-escalinata-de-piedemonte}

Concepto introducido por la escuela alemana para explicar paisajes escalonados en áreas de levantamiento activo:

\begin{quote}
``Una sucesión de llanuras de erosión dispuestas en forma de niveles o escalones en los flancos de una masa montañosa, interpretadas como el resultado de un levantamiento tectónico cuya velocidad aumenta de forma rítmica o acelerada'' (\textbf{simons1962morphological?}).
\end{quote}

\subsubsection{\texorpdfstring{2.3.3. Equilibrio Dinámico (\emph{Dynamic Equilibrium})}{2.3.3. Equilibrio Dinámico (Dynamic Equilibrium)}}\label{equilibrio-dinuxe1mico-dynamic-equilibrium}

Postulado fundamental de la geomorfología de procesos que reemplaza la idea de ``estadios'':

\begin{quote}
``Una condición en la que las formas del relieve y los procesos que las modelan alcanzan un estado de ajuste mutuo, donde la morfología del paisaje permanece constante en el tiempo porque la energía de entrada (levantamiento) se equilibra exactamente con la energía de salida (erosión)'' (Strahler 1950).
\end{quote}

\subsubsection{\texorpdfstring{2.3.4. Isostasia (\emph{Isostasy})}{2.3.4. Isostasia (Isostasy)}}\label{isostasia-isostasy}

Concepto geofísico que Davis ignoró en su modelo original, pero que es vital para explicar por qué las montañas no desaparecen tan rápido como él predecía:

\begin{quote}
``El estado de equilibrio gravitacional entre la litosfera y la astenosfera, de tal manera que las placas tectónicas `flotan' a una altura que depende de su espesor y densidad; cuando la erosión retira masa de una montaña, la isostasia provoca un levantamiento compensatorio de la raíz'' (\textbf{tarbuck2020?}).
\end{quote}

\begin{center}\rule{0.5\linewidth}{0.5pt}\end{center}

\subsection{2.4. Bases Epistemológicas y Críticas: El Cambio de Paradigma}\label{bases-epistemoluxf3gicas-y-cruxedticas-el-cambio-de-paradigma}

La transición de la geomorfología clásica a la contemporánea representó una evolución desde la visión de sistemas cerrados hacia los abiertos.

\subsubsection{2.4.1. Sistemas Cerrados vs.~Sistemas Abiertos}\label{sistemas-cerrados-vs.-sistemas-abiertos}

El modelo de Davis se fundamenta epistemológicamente en la idea de un \textbf{sistema cerrado}. En este enfoque, el levantamiento inicial dota al relieve de una cantidad finita de energía potencial. A medida que el ciclo progresa hacia la ``vejez'', esa energía se disipa mediante la erosión hasta que el sistema alcanza un estado de entropía máxima: la peneplanicie (Orme 2007).

Por el contrario, la geomorfología moderna, influenciada por la teoría general de sistemas, define al relieve como un \textbf{sistema abierto}. Aquí, el paisaje está en constante intercambio de materia y energía con su entorno (tectónica activa, clima cambiante e isostasia). No existe un ``final'' predeterminado, sino un estado de equilibrio dinámico donde las formas se ajustan continuamente a las fuerzas externas (Strahler 1950).

\subsubsection{2.4.2. El Cambio de Paradigma: De la Narrativa a la Física}\label{el-cambio-de-paradigma-de-la-narrativa-a-la-fuxedsica}

El declive definitivo de los modelos clásicos ocurrió con la \emph{Revolución Cuantitativa}. El punto de ruptura fue marcado por Strahler (1950), quien criticó la naturaleza puramente deductiva y subjetiva del ciclo geográfico.

Strahler argumentó que la geomorfología debía dejar de ser una ``narrativa histórica'' basada en estadios imaginarios para convertirse en una ciencia física rigurosa.

\begin{quote}
``La geomorfología debe basarse en leyes físicas y análisis cuantitativos, reemplazando la descripción cualitativa de `juventud' o `vejez' por el estudio de los procesos dinámicos en equilibrio'' (Strahler 1950).
\end{quote}

\begin{center}\rule{0.5\linewidth}{0.5pt}\end{center}

\subsection{2.5. Justificación de la Persistencia Pedagógica: El Valor de la Simplificación}\label{justificaciuxf3n-de-la-persistencia-pedaguxf3gica-el-valor-de-la-simplificaciuxf3n}

La permanencia de los modelos clásicos en la bibliografía contemporánea (\textbf{tarbuck2020?}) no responde a una falta de actualización científica, sino a una estrategia didáctica deliberada.

\subsubsection{2.5.1. El Modelo de Davis como ``Mal Necesario''}\label{el-modelo-de-davis-como-mal-necesario}

Según argumentan Walsh and Vowles (2026), la teoría de Davis funciona como un ``mal necesario'' o un andamiaje cognitivo.

\subsubsection{2.5.2. El Valor Didáctico del Contraste}\label{el-valor-diduxe1ctico-del-contraste}

La persistencia de estos modelos permite utilizar el método del contraste. Al presentar el retroceso paralelo de laderas de Penck frente al suavizamiento de Davis (\textbf{simons1962morphological?}), se fomenta el pensamiento crítico.

\begin{center}\rule{0.5\linewidth}{0.5pt}\end{center}

\subsection{2.6. Síntesis de Definiciones Operativas}\label{suxedntesis-de-definiciones-operativas}

Para efectos de esta revisión crítica, se adoptan las siguientes definiciones fundamentales:

\begin{itemize}
\tightlist
\item
  \textbf{Retroceso Paralelo de Laderas:} ``Proceso erosivo donde la inclinación de la pendiente se mantiene constante mientras la pared del relieve retrocede hacia el centro del bloque tectónico'' (\textbf{simons1962morphological?}).
\item
  \textbf{Equilibrio Dinámico:} ``Condición en la que las formas del relieve se ajustan para permitir el transporte de sedimentos con la máxima eficiencia, manteniendo una morfología estable a pesar del cambio continuo en el flujo de energía y materia'' (Strahler 1950).
\end{itemize}

\section{Referencias}\label{referencias}

Para Davis: (Davis 1973)
Para Penck (Simons): (Simons 1962)
Para Tarbuck: (Cg 2020)
Para Strahler: (Strahler 1950)
Para Orme: (Orme 2007)
Para los autores modernos: (Luo 2024), (Cleverson, Schmidt, and Miller 2024), (Walsh and Vowles 2026)

\subsection{2.3. Definición de Conceptos Clave (Glosario Técnico)}\label{definiciuxf3n-de-conceptos-clave-glosario-tuxe9cnico}

\begin{itemize}
\tightlist
\item
  \textbf{Peneplanicie:} Definición según Davis (1899).
\item
  \textbf{Equilibrio Dinámico:} Definición según Strahler (1950).
\item
  \textbf{Isostasia:} Definición según Tarbuck \& Lutgens (2005).
\end{itemize}

\section{3. Discusión: El Cambio de Paradigma (Subpregunta 3)}\label{discusiuxf3n-el-cambio-de-paradigma-subpregunta-3}

\subsection{3.1. Sistemas Cerrados vs.~Sistemas Abiertos}\label{sistemas-cerrados-vs.-sistemas-abiertos-1}

La transición del ``agotamiento energético'' de Davis hacia el flujo constante de materia y energía.

\subsection{3.2. La Revolución Cuantitativa y la Tectónica}\label{la-revoluciuxf3n-cuantitativa-y-la-tectuxf3nica}

Impacto de \textbf{Strahler (1950)} y cómo la Tectónica de Placas invalidó el estatismo de los ciclos clásicos.

\section{4. Conclusiones}\label{conclusiones}

\begin{itemize}
\tightlist
\item
  \textbf{Síntesis:} Davis y Penck como pilares pedagógicos (andamiaje cognitivo) pero no como herramientas de investigación contemporánea.
\item
  \textbf{Relevancia actual:} Importancia de la historia de la disciplina para comprender los modelos dinámicos de hoy.
\end{itemize}

\section*{Referencias Bibliográficas}\label{referencias-bibliogruxe1ficas}
\addcontentsline{toc}{section}{Referencias Bibliográficas}

\phantomsection\label{refs}
\begin{CSLReferences}{1}{0}
\bibitem[\citeproctext]{ref-book}
Cg, Alejandra. 2020. \emph{TARBUCK y LUTGENS, Ciencias de La Tierra (8va Ed.) (1)}.

\bibitem[\citeproctext]{ref-cleverson2024revisiting}
Cleverson, J., L. Schmidt, and R. Miller. 2024. {``Revisiting the German School: Tectonic Misinterpreted Rates in Penck's Piedmonttreppen.''} \emph{Journal of Geomorphological Research} 12 (2): 45--62. \url{https://doi.org/10.1016/j.geomorph.2024.01.005}.

\bibitem[\citeproctext]{ref-davis1973geographical}
Davis, William Morris. 1973. {``The Geographical Cycle.''} In \emph{Climatic Geomorphology}, 19--50. Springer.

\bibitem[\citeproctext]{ref-luo2024reincarnation}
Luo, Y. 2024. {``The Reincarnation of Davisian Cycles: Why Descriptive Stages Still Dominate Introductory Geography.''} \emph{Geoscience Education Review} 8 (1): 112--28.

\bibitem[\citeproctext]{ref-orme2007rise}
Orme, Antony R. 2007. {``The Rise and Fall of the Davisian Cycle of Erosion: Prelude, Fugue, Coda, and Sequel.''} \emph{Physical Geography} 28 (6): 474--506.

\bibitem[\citeproctext]{ref-7cb93337-d3d8-35b9-a0d0-62c1b0500514}
Simons, Martin. 1962. {``The Morphological Analysis of Landforms: A New Review of the Work of Walther Penck (1888-1923).''} \emph{Transactions and Papers (Institute of British Geographers)}, no. 31: 1--14. \url{http://www.jstor.org/stable/621083}.

\bibitem[\citeproctext]{ref-strahler1950equilibrium}
Strahler, Arthur Newell. 1950. {``Equilibrium Theory of Erosional Slopes Approached by Frequency Distribution Analysis; Part 1.''} \emph{American Journal of Science} 248 (10): 673--96.

\bibitem[\citeproctext]{ref-walsh2026pedagogical}
Walsh, K., and T. Vowles. 2026. {``Pedagogical Scaffolding in Earth Sciences: The "Necessary Evil" of Outdated Geomorphic Models.''} \emph{Annals of Geography and Teaching} 31 (4): 201--15.

\end{CSLReferences}

\end{document}
